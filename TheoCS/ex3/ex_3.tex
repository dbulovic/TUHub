\documentclass{article}
\usepackage{amsmath}
\usepackage{amssymb}

\title {Theoretical Computer Science: Exercise 3 - Reductions}
\author {David Bulovic}
\date {}

\begin{document}
\maketitle

Show that the language: \\
\begin{math} X = \{ \langle M,w \rangle | \end{math} 
M is a 1-band TM and M does not modify the part of the band where the input w is\} 
is not decidable. \\

Let us assume that X is decidable and let R be the Turing machine that decides X. Now, we can create a Turing machine 
S that decides A\textsubscript{TM}\footnote{A\textsubscript{TM} \begin{math} = \{ \langle T,w \rangle | \end{math} T is a TM and T accepts w\}}.
The construction of S is the following:
\begin{itemize}
\item The input is \begin{math}\langle M,w \rangle\end{math}, M being a Turing Machine and w being a string.
\item Now, we create a new Turing Machine M' using the following steps:
\begin{itemize}
\item The input is code y
\item Run M with input w, without editing the input y
\item If M accepts, M' accepts
\item Otherwise delete the input string y and write 1 and reject
\end{itemize}
\item Now run R with M' as input
\item S accepts if R accepts, S rejects if R rejects
\end{itemize}
If M accepts w, M' will not modify the input (which means it will be accepted by R). If M does not accept w, M' will modify the input
(it will be rejected by R). This means that R will accept M' if and only if M accepts w. Consequently, S decides A\textsubscript{TM}.\\

However, since we know that A\textsubscript{TM} is undecidable, S cannot exists. 
Due to this contradiction R cannot exist either, so X must be undecidable.


\end{document}